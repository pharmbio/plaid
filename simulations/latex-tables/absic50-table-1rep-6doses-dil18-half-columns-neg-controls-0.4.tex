 & Border & Random & Effective \\ 
\hline
\tabCount{} & 19177.0 & 19200.0 & 19200.0\\ 
\tabMean{} & inf & 0.2 & 0.17\\ 
\tabSTD{} & nan & 0.19 & 0.15\\ 
\tabMin{} & 0.0 & 0.0 & 0.0\\ 
\tabQone{} & 0.08 & 0.07 & 0.06\\ 
\tabMedian{} & 0.25 & 0.14 & 0.13\\ 
\tabQthree{} & 1.14 & 0.27 & 0.24\\ 
\tabMax{} & inf & 1.72 & 1.73\\ 
\hline